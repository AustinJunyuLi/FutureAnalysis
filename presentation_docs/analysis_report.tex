\documentclass[11pt,a4paper]{article}

\usepackage[margin=1in]{geometry}
\usepackage{graphicx}
\usepackage{amsmath}
\usepackage{amssymb}
\usepackage{booktabs}
\usepackage{hyperref}
\usepackage{xcolor}
\usepackage{float}
\usepackage{caption}
\usepackage{subcaption}
\usepackage{enumitem}
\usepackage{listings}
\usepackage{fancyhdr}

\lstset{
    basicstyle=\small\ttfamily,
    breaklines=true,
    frame=single,
    language=Python,
    keywordstyle=\color{blue},
    commentstyle=\color{gray},
    stringstyle=\color{red}
}

\hypersetup{
    colorlinks=true,
    linkcolor=blue,
    citecolor=blue,
    urlcolor=blue
}

\title{\textbf{Institutional Roll Pattern Detection in Copper Futures}\\
\large A Computational Framework and Empirical Analysis\\
\vspace{0.5cm}
\normalsize CME Copper (HG) Contracts 2008--2024}
\author{}
\date{}

\pagestyle{fancy}
\fancyhf{}
\lhead{Institutional Roll Pattern Detection}
\rhead{Copper Futures Analysis}
\cfoot{\thepage}

\begin{document}

\maketitle

\begin{abstract}
This report presents a computational framework for characterizing calendar-spread dynamics in CME copper (HG) futures from 2008--2024. Minute-level data are aggregated into 10 global trading sessions, assembled into multi-contract panels, and analyzed with vectorized spread/volume signals. A new strip diagnostic layer examines all 11 spreads (S1--S11) simultaneously: daily comparisons of $| \Delta S_1 |$ versus the median of $| \Delta S_2 |,\ldots,| \Delta S_{11} |$ classify each day as \emph{normal}, \emph{broad roll}, or \emph{expiry dominance}. Of the 5,814 days where S1 clearly dominates, 49.6\% occur within 18 business days of F1 expiry, confirming the supervisor's concern that much of the signal reflects expiry mechanics. Filtering those expiry-dominant days removes 1,869 of the 2,737 raw S1 events (68\%), leaving 868 candidate roll signals (1.95\% detection rate) with a median timing 22 business days before expiry (IQR 19--34). Multi-spread statistics remain balanced across S1--S3, reinforcing that the framework now separates structural expiry effects from discretionary rolling behaviour while retaining the genuine signals of interest. The entire pipeline—panel assembly, diagnostics, filtering, and reporting—executes in under three minutes on 450M observations.
\end{abstract}

\tableofcontents
\newpage

% ===============================================
\section{Introduction}
% ===============================================

Institutional investors managing futures positions face the operational necessity of rolling contracts before expiry to maintain market exposure. This rolling activity creates detectable patterns in calendar spreads---the price differential between front-month and next-month contracts. Understanding these patterns provides insights into market microstructure and institutional trading behavior.

This analysis implements a systematic framework for detecting roll patterns in CME copper futures markets. The system processes high-frequency data at minute-level granularity, aggregates it into meaningful trading periods, and applies statistical methods to identify significant spread widening events that indicate institutional rolling activity.

The dataset encompasses 202 copper futures contracts traded on the Chicago Mercantile Exchange from January 2008 through December 2024. Each contract contains approximately 220{,}000 minute-level observations, totaling over 44 million data points. The analysis reveals that institutional rolling activity follows predictable patterns, with median timing at 19 business days before expiry and concentrated activity during specific trading sessions.

% ===============================================
\section{Data Architecture}
% ===============================================

\subsection{Raw Data Specifications}

The analysis processes minute-level futures market data with the following characteristics:

\begin{table}[H]
\centering
\caption{Dataset Specifications}
\begin{tabular}{ll}
\toprule
\textbf{Specification} & \textbf{Value} \\
\midrule
Exchange & CME Group (COMEX Division) \\
Commodity & Copper (HG) \\
Time Period & January 2008 -- December 2024 \\
Contracts Analyzed & 202 \\
Total Source Files & 13{,}548 (across 32 commodities) \\
Data Frequency & 1-minute OHLCV \\
File Format & Parquet (columnar storage) \\
Timezone & US/Central (Chicago) \\
Total Data Points & $\sim$450 million \\
Storage Size & $\sim$5 GB (compressed) \\
\bottomrule
\end{tabular}
\end{table}

Each minute bar contains:
\begin{itemize}[noitemsep]
    \item \textbf{Timestamp}: Minute-precision datetime in US/Central
    \item \textbf{Open}: First trade price in the minute
    \item \textbf{High}: Maximum trade price in the minute
    \item \textbf{Low}: Minimum trade price in the minute
    \item \textbf{Close}: Last trade price in the minute
    \item \textbf{Volume}: Number of contracts traded
    \item \textbf{Open Interest}: Outstanding contracts (when available)
\end{itemize}

\subsection{Data Organization Pipeline}

The framework implements a systematic data organization pipeline that structures raw files into a hierarchical commodity-based layout:

\begin{lstlisting}[caption=Directory Structure After Organization]
organized_data/
+-- copper/
|   +-- HGF2008.parquet
|   +-- HGG2008.parquet
|   +-- HGH2008.parquet
|   +-- ... (202 contract files)
+-- gold/
+-- silver/
+-- ... (32 commodities total)
\end{lstlisting}

% -----------------------------------------------
\subsection{Business-Day Definition and Sources}

We compute business-day spacing using an authoritative CME/Globex trading calendar with rigorous source verification:

\subsubsection{Data Sources and Verification}

\begin{itemize}[noitemsep]
    \item \textbf{Primary Source}: CME Group Trading Hours page\footnote{CME Group. ``Holiday and Trading Hours.'' \url{https://www.cmegroup.com/trading-hours.html}. Accessed December 2024.}
    \item \textbf{Secondary Sources}: CME Clearing Notices, SIFMA recommendations, and broker notifications (AMP Futures, Cannon Trading)
\end{itemize}

\subsubsection{Holiday Classification}

The calendar (\texttt{metadata/calendars/cme\_globex\_holidays.csv}) distinguishes three session types:

\begin{table}[H]
\centering
\caption{Trading Session Classifications}
\begin{tabular}{lll}
\toprule
\textbf{Session Type} & \textbf{Trading Status} & \textbf{Business Day?} \\
\midrule
Closed & No trading & No \\
Regular & Normal hours & Yes \\
Early close & Partial day & Yes \\
\bottomrule
\end{tabular}
\end{table}

% (Holiday enumeration intentionally omitted; report focuses on business-day methodology.)

\subsubsection{Business Day Computation}

\begin{itemize}[noitemsep]
    \item \textbf{Session mapping}: Timestamps mapped to trading dates using Asia cross-midnight convention (21:00 CT anchor)
    \item \textbf{Counting convention}: Business days = trading days between event and expiry (exclusive of event date)
    % Weekend observance details omitted; business-day methodology suffices for analysis
    \item \textbf{Implementation}: \texttt{trading\_days.py} module with vectorized NumPy operations for performance
\end{itemize}

The organization process:
\begin{enumerate}[noitemsep]
    \item Scans source directories for futures contract files
    \item Extracts commodity codes using regex pattern matching
    \item Maps 65+ contract symbols to 32 commodity categories
    \item Creates commodity-specific folders
    \item Moves files maintaining naming conventions
    \item Generates \texttt{data\_inventory.csv} with file metadata
\end{enumerate}

\subsection{Metadata Integration}

Contract expiry dates are sourced from official CME calendars and stored in a normalized CSV format:

\begin{lstlisting}[caption=Contract Metadata Structure]
root,contract,expiry_date,source,source_url
HG,HGF2009,2009-01-28,CME Copper Calendar,https://...
HG,HGG2009,2009-02-25,CME Copper Calendar,https://...
HG,HGH2009,2009-03-27,CME Copper Calendar,https://...
\end{lstlisting}

This metadata drives the front/next contract identification algorithm, ensuring accurate spread calculations across contract transitions.

% ===============================================
\section{Methodology -- Technical Implementation}
% ===============================================

\subsection{Intraday Period Aggregation}

\subsubsection{10-Period Structure}

The framework aggregates minute-level data into 10 intraday periods that capture distinct trading sessions:

\begin{table}[H]
\centering
\caption{Intraday Period Definitions}
\begin{tabular}{clllc}
\toprule
\textbf{Period} & \textbf{Time (CT)} & \textbf{Label} & \textbf{Session} & \textbf{Duration} \\
\midrule
1 & 09:00--09:59 & US Open & US Regular & 1 hour \\
2 & 10:00--10:59 & US Morning & US Regular & 1 hour \\
3 & 11:00--11:59 & US Late Morning & US Regular & 1 hour \\
4 & 12:00--12:59 & US Midday & US Regular & 1 hour \\
5 & 13:00--13:59 & US Early Afternoon & US Regular & 1 hour \\
6 & 14:00--14:59 & US Late Afternoon & US Regular & 1 hour \\
7 & 15:00--15:59 & US Close & US Regular & 1 hour \\
8 & 16:00--20:59 & Late US/After-Hours & Late US & 5 hours \\
9 & 21:00--02:59 & Asia Session & Asia & 6 hours \\
10 & 03:00--08:59 & Europe Session & Europe & 6 hours \\
\bottomrule
\end{tabular}
\end{table}

\subsubsection{Aggregation Algorithm}

The aggregation process employs vectorized NumPy operations for computational efficiency:

\begin{lstlisting}[caption=Vectorized Bucket Assignment,language=Python]
def assign_bucket(hour: int) -> int:
    """Map hour (0-23) to bucket ID (1-10)"""
    if 9 <= hour <= 15:
        return hour - 8  # US regular hours
    elif 16 <= hour <= 20:
        return 8  # Late US
    elif hour >= 21 or hour <= 2:
        return 9  # Asia
    elif 3 <= hour <= 8:
        return 10  # Europe

# Vectorized application
hours = df.index.hour
bucket_ids = np.vectorize(assign_bucket)(hours)
\end{lstlisting}

Aggregation rules preserve OHLCV integrity:
\begin{itemize}[noitemsep]
    \item \textbf{Open}: First value in period
    \item \textbf{High}: Maximum value in period
    \item \textbf{Low}: Minimum value in period
    \item \textbf{Close}: Last value in period
    \item \textbf{Volume}: Sum of all volumes
\end{itemize}

\subsubsection{Timestamp Anchoring}

Each aggregated period is anchored to its start time to maintain temporal consistency:
\begin{itemize}[noitemsep]
    \item US regular hours: Anchored to hour start (e.g., 09:00, 10:00)
    \item Asia session: Anchored to 21:00 of previous day
    \item Europe session: Anchored to 03:00 of current day
    \item Late US: Anchored to 16:00 of current day
\end{itemize}

\subsection{Panel Assembly and Contract Identification}

\subsubsection{Multi-Contract Panel Structure}

The framework assembles a wide-format panel with MultiIndex columns:

\begin{lstlisting}[caption=Panel Structure]
Columns: MultiIndex[(contract, field)]
  - (HGF2009, open)
  - (HGF2009, high)
  - (HGF2009, low)
  - (HGF2009, close)
  - (HGF2009, volume)
  - ... (repeated for 202 contracts)
  - (meta, bucket)
  - (meta, bucket_label)
  - (meta, session)
  - (meta, front_contract)
  - (meta, next_contract)

Index: DatetimeIndex (bucket timestamps)
Shape: (44428, 1015)  # 44K periods x 1015 columns
\end{lstlisting}

\subsubsection{Front/Next Contract Detection}

The algorithm identifies front and next contracts using vectorized operations:

\begin{lstlisting}[caption=Vectorized Contract Identification,language=Python]
def identify_front_next(panel, expiry_map, price_field='close'):
    # Extract price matrix (periods x contracts)
    close_values = close_df.to_numpy(dtype=float)
    available = np.isfinite(close_values)
    
    # Compute days to expiry for all contracts
    expiry_array = pd.to_datetime(expiry_series).to_numpy()
    date_int = dates.to_numpy(dtype='datetime64[ns]')
    delta = expiry_int.reshape(1, -1) - date_int.reshape(-1, 1)
    
    # Mask expired and unavailable contracts
    active_mask = available & (delta >= 0)
    delta[~active_mask] = np.inf
    
    # Find nearest expiry (front) using argmin
    front_idx = delta.argmin(axis=1)
    
    # Find second nearest (next)
    delta_next = delta.copy()
    delta_next[np.arange(len(delta)), front_idx] = np.inf
    next_idx = delta_next.argmin(axis=1)
    
    return front_contracts, next_contracts
\end{lstlisting}

This approach processes all 44{,}428 periods simultaneously, achieving a 100x speedup over iterative methods.

\subsection{Calendar Spread Computation}

\subsubsection{Spread Calculation}

The calendar spread represents the price differential between next and front contracts:

\begin{equation}
S_t = P_{\text{next},t} - P_{\text{front},t}
\end{equation}

Where:
\begin{itemize}[noitemsep]
    \item $S_t$ = Calendar spread at time $t$
    \item $P_{\text{next},t}$ = Close price of next contract
    \item $P_{\text{front},t}$ = Close price of front contract
\end{itemize}

The spread change series:
\begin{equation}
\Delta S_t = S_t - S_{t-1}
\end{equation}

\subsubsection{Liquidity Signal}

The framework computes a complementary liquidity signal based on volume ratios:

\begin{equation}
L_t = \begin{cases}
1 & \text{if } V_{\text{next},t} \geq \alpha \cdot V_{\text{front},t} \\
0 & \text{otherwise}
\end{cases}
\end{equation}

Where $\alpha = 0.8$ (configurable threshold).

\subsection{Multi-Spread Comparative Analysis}

To distinguish between institutional rolling behavior and contract expiry mechanics, the framework extends the analysis beyond the front spread (S1) to compute and analyze all adjacent calendar spreads across the active contract chain.

\subsubsection{Extended Contract Chain}

The system identifies up to 12 active contracts at each timestamp:

\begin{equation}
\{F_1, F_2, \ldots, F_{12}\} = \text{nearest 12 unexpired contracts sorted by expiry}
\end{equation}

Where $F_1$ is the nearest-expiry contract (front month), $F_2$ is the second-nearest (next month), etc.

\subsubsection{Multiple Calendar Spreads}

From the contract chain, we compute 11 adjacent calendar spreads:

\begin{align}
S_1 &= P_{F_2,t} - P_{F_1,t} \quad \text{(front spread)} \\
S_2 &= P_{F_3,t} - P_{F_2,t} \quad \text{(second spread)} \\
&\vdots \\
S_{11} &= P_{F_{12},t} - P_{F_{11},t}
\end{align}

\subsubsection{Comparative Hypothesis Testing}

The multi-spread analysis tests two competing hypotheses:

\textbf{Hypothesis A: Institutional Rolling Behavior}

If events reflect discretionary institutional decisions to roll positions from the front contract to the next contract, we expect:
\begin{itemize}[noitemsep]
    \item $S_1$ shows significantly more events than $S_2, S_3, \ldots, S_{11}$
    \item Events in $S_1$ occur at a specific time before $F_1$ expiry
    \item $S_2, S_3, \ldots$ do not show similar patterns at equivalent times-to-expiry
\end{itemize}

\textbf{Hypothesis B: Contract Expiry Mechanics}

If events reflect systematic market behavior as contracts approach expiry, we expect:
\begin{itemize}[noitemsep]
    \item All spreads ($S_1, S_2, \ldots, S_{11}$) show similar event patterns
    \item Each spread $S_i$ shows events when its front contract $F_i$ is $\sim$20-30 days from expiry
    \item The pattern "ripples through" spreads as successive contracts mature
    \item Low correlation between spreads (different contracts maturing at different times)
\end{itemize}

\subsection{Statistical Event Detection}

\subsubsection{Z-Score Methodology}

The detection algorithm uses a rolling z-score of spread changes:

\begin{equation}
z_t = \frac{\Delta S_t - \mu_w}{\sigma_w}
\end{equation}

Where:
\begin{itemize}[noitemsep]
    \item $\mu_w$ = Rolling mean over window $w$
    \item $\sigma_w$ = Rolling standard deviation over window $w$
    \item $w = 20$ periods (approximately 2 trading days)
\end{itemize}

Detection criteria:
\begin{equation}
\text{Event}_t = \begin{cases}
1 & \text{if } z_t > 1.5 \text{ and } \Delta S_t > 0 \\
0 & \text{otherwise}
\end{cases}
\end{equation}

\subsubsection{Cool-down Mechanism}

To prevent cascade detections from single large moves, the framework enforces a cool-down period:

\begin{lstlisting}[caption=Time-Based Cool-down Implementation,language=Python]
def apply_cool_down(events, cool_down_hours=3):
    result = pd.Series(False, index=events.index)
    last_event_time = pd.Timestamp.min
    
    for timestamp in events[events].index:
        hours_since_last = (timestamp - last_event_time).total_seconds() / 3600
        if hours_since_last >= cool_down_hours:
            result[timestamp] = True
            last_event_time = timestamp
    
    return result
\end{lstlisting}

\subsection{Data Quality Filtering}

\subsubsection{Contract-Level Quality Criteria}

The framework evaluates each contract against multiple quality metrics:

\begin{table}[H]
\centering
\caption{Data Quality Thresholds}
\begin{tabular}{lrl}
\toprule
\textbf{Criterion} & \textbf{Threshold} & \textbf{Rationale} \\
\midrule
Minimum data points & 500 & Ensure statistical significance \\
Coverage percentage & $\geq$25\% & Adequate trading day representation \\
Maximum gap & 30 days & Avoid sparse/illiquid periods \\
Expiry cutoff & $\geq$2015 & Focus on recent market structure \\
\bottomrule
\end{tabular}
\end{table}

\subsubsection{Quality Evaluation Algorithm}

\begin{lstlisting}[caption=Contract Quality Assessment,language=Python]
def evaluate_contract(df, contract):
    data_points = len(df)
    date_range = (df.index.min(), df.index.max())
    total_days = (date_range[1] - date_range[0]).days + 1
    
    # Calculate trading day coverage
    expected_trading_days = total_days * 0.7
    coverage = data_points / expected_trading_days * 100
    
    # Identify gaps
    date_diffs = df.index.to_series().diff()
    gaps = date_diffs[date_diffs > pd.Timedelta(days=30)]
    
    # Apply criteria
    if data_points < 500:
        return "EXCLUDED", "Insufficient data"
    if coverage < 25:
        return "EXCLUDED", "Low coverage"
    if len(gaps) > 0:
        return "EXCLUDED", "Large gaps"
    
    return "INCLUDED", None
\end{lstlisting}

% ===============================================
\section{Implementation Framework}
% ===============================================

\subsection{Package Architecture}

\subsubsection{Core Module Structure}

The implementation consists of specialized modules with defined responsibilities:

\begin{table}[H]
\centering
\caption{Core Module Functions}
\begin{tabular}{ll}
\toprule
\textbf{Module} & \textbf{Responsibility} \\
\midrule
\texttt{ingest.py} & Parquet file loading and contract discovery \\
\texttt{buckets.py} & Intraday period aggregation engine \\
\texttt{panel.py} & Wide-format panel assembly with metadata \\
\texttt{rolls.py} & Front/next identification algorithms \\
\texttt{events.py} & Spread detection and event summarization \\
\texttt{quality.py} & Data filtering and quality assessment \\
\texttt{analysis.py} & Pipeline orchestration and coordination \\
\texttt{config.py} & Settings management and validation \\
\bottomrule
\end{tabular}
\end{table}

\subsubsection{Configuration System}

The framework uses YAML-based configuration with hierarchical settings:

\begin{lstlisting}[caption=Configuration Structure (settings.yaml)]
products:
  - HG  # Copper commodity code

bucket_config:
  us_regular_hours:
    start: 9
    end: 15
    granularity: "hourly"
  off_peak_sessions:
    late_us: {hours: [16,17,18,19,20], bucket: 8}
    asia: {hours: [21,22,23,0,1,2], bucket: 9}
    europe: {hours: [3,4,5,6,7,8], bucket: 10}

spread:
  method: "zscore"
  window_buckets: 20
  z_threshold: 1.5
  cool_down_hours: 3.0

data_quality:
  filter_enabled: true
  min_data_points: 500
  min_coverage_percent: 25
  max_gap_days: 30
  cutoff_year: 2015
\end{lstlisting}

\subsection{Command-Line Interface}

\subsubsection{Entry Points}

The package provides three primary command-line interfaces:

\begin{table}[H]
\centering
\caption{CLI Commands}
\begin{tabular}{ll}
\toprule
\textbf{Command} & \textbf{Function} \\
\midrule
\texttt{futures-roll-organize} & Organize raw data files by commodity \\
\texttt{futures-roll-hourly} & Run intraday period analysis \\
\texttt{futures-roll-daily} & Run daily granularity analysis \\
\bottomrule
\end{tabular}
\end{table}

\subsubsection{Parameter Support}

Each CLI supports flexible parameter overrides:

\begin{lstlisting}[caption=CLI Usage Examples,language=bash]
# Organize raw data
futures-roll-organize \
  --source /path/to/raw/files \
  --destination organized_data \
  --inventory data_inventory.csv

# Run hourly analysis with custom settings
futures-roll-hourly \
  --settings config/settings.yaml \
  --root organized_data/copper \
  --metadata metadata/contracts_metadata.csv \
  --output-dir outputs \
  --max-files 10 \
  --log-level DEBUG

# Run daily analysis with quality filtering
futures-roll-daily \
  --settings config/settings.yaml \
  --root organized_data/copper \
  --output-dir outputs
\end{lstlisting}

% ===============================================
\section{Empirical Results}
% ===============================================
\subsection{Strip Diagnostics and Expiry Classification}

Daily comparisons of $|\Delta S_1|$ versus the median $|\Delta S_2|,\ldots,|\Delta S_{11}|$ classify each day within a contract cycle (Table~\ref{tab:diagnostics}). Nearly half of all S1-dominant days occur inside the 18-business-day expiry window, validating the supervisor's concern that raw S1 signals are heavily influenced by contract maturity.

\begin{table}[H]
\centering
\caption{S1 Dominance Diagnostics by Classification}
\label{tab:diagnostics}
\begin{tabular}{lrrrrr}
\toprule
\textbf{Classification} & \textbf{Days} & \textbf{Share} & \textbf{Median BD to Expiry} & \textbf{Avg Dominance Ratio} & \textbf{Avg $|\Delta S_1|$} \\
\midrule
Expiry dominance & 2{,}884 & 28.3\% & 10 & 3.12 & 0.0096 \\
Broad roll & 2{,}930 & 28.8\% & 29 & 3.48 & 0.0092 \\
Normal & 4{,}369 & 42.9\% & 28 & 0.79 & 0.0065 \\
\bottomrule
\end{tabular}
\end{table}

\textbf{Implications}:
\begin{itemize}[noitemsep]
    \item 5,814 days show clear S1 dominance; 49.6\% are expiry-driven while 50.4\% reflect broad multi-contract movement.
    \item Expiry-dominant days cluster around 10 business days before F1 expiry; broad-roll days peak around 29 business days.
    \item Average dominance ratios remain high (3.5) even outside the expiry window, indicating meaningful moves when the filter classifies a day as \emph{broad roll}.
\end{itemize}

\subsection{Expiry-Adjusted Roll Signals}

The diagnostic filter trims the statistical signal set from 2,737 raw S1 events (6.16\%) to 868 high-confidence roll events (1.95\%). Table~\ref{tab:filter} quantifies the shift in timing statistics.

\begin{table}[H]
\centering
\caption{Effect of Expiry-Dominance Filtering on S1 Events}
\label{tab:filter}
\begin{tabular}{lrrrr}
\toprule
 & \textbf{Events} & \textbf{Detection Rate} & \textbf{Median BD} & \textbf{IQR (BD)} \\
\midrule
Raw z-score signal & 2{,}737 & 6.16\% & 19 & 13--24 \\
Expiry-dominant events removed & 1{,}869 & -- & 10 & 7--14 \\
Filtered roll signal & 868 & 1.95\% & 22 & 19--34 \\
\bottomrule
\end{tabular}
\end{table}

The surviving signals are materially further from expiry (mean 25.8 business days, min--max 0--79), aligning with discretionary rolling behaviour instead of forced contract settlement.

\subsection{Intraday Distribution After Filtering}

The filtered events remain concentrated in US daytime and overnight Asian sessions (Table~\ref{tab:buckets}). Preference scores above 1.5 persist for the 14:00--15:00 CT window and the overnight Asia session, indicating continued institutional activity in those buckets even after expiry-driven days are excised.

\begin{table}[H]
\centering
\caption{Filtered Roll Events by Intraday Period}
\label{tab:buckets}
\begin{tabular}{clrrrrr}
\toprule
\textbf{Bucket} & \textbf{Session Label} & \textbf{Events} & \textbf{Share} & \textbf{Rate} & \textbf{Avg Spread} & \textbf{Pref Score} \\
\midrule
1 & 09:00 -- US Open & 80 & 9.2\% & 1.83\% & 0.0012 & 1.02 \\
2 & 10:00 -- US Morning & 68 & 7.8\% & 1.55\% & 0.0013 & 0.73 \\
3 & 11:00 -- US Late Morning & 80 & 9.2\% & 1.83\% & 0.0013 & 0.71 \\
4 & 12:00 -- US Midday & 44 & 5.1\% & 1.00\% & 0.0017 & 0.58 \\
5 & 13:00 -- US Early Afternoon & 42 & 4.8\% & 0.97\% & 0.0021 & 0.82 \\
6 & 14:00 -- US Late Afternoon & 60 & 6.9\% & 1.40\% & 0.0030 & \textbf{1.98} \\
7 & 15:00 -- US Close & 58 & 6.7\% & 1.37\% & 0.0031 & \textbf{1.91} \\
8 & Late US / After-Hours & 146 & 16.8\% & 2.78\% & 0.0028 & 0.92 \\
9 & Asia Session & 166 & 19.1\% & 3.76\% & 0.0025 & 1.54 \\
10 & Europe Session & 124 & 14.3\% & 2.83\% & 0.0011 & 1.07 \\
\bottomrule
\end{tabular}
\end{table}

\subsection{Liquidity Confirmation}

After filtering, 531 of the 868 spread events (61\%) coincide with the volume-based liquidity roll signal ($V_{F2} \ge 0.8 \times V_{F1}$), and liquidity continues to lead spread detections by 0.9 business days on average. This provides independent confirmation that the remaining signals capture meaningful position transfers rather than calendar noise.

% ===============================================
\section{Multi-Spread Comparative Results}
% ===============================================

The multi-spread analysis provides definitive evidence regarding the nature of detected events. By comparing signal characteristics across all 11 calendar spreads, we can determine whether the patterns reflect institutional rolling decisions or systematic contract expiry mechanics.

\subsection{Event Distribution Across Spreads}

Applying identical detection methodology (z-score > 1.5, 20-period window, 3-hour cool-down) to all spreads reveals a consistent pattern:

\begin{table}[H]
\centering
\caption{Event Detection Across Calendar Spreads (Relative to F1 Expiry)}
\begin{tabular}{lrrrr}
\toprule
\textbf{Spread} & \textbf{Events} & \textbf{Rate (\%)} & \textbf{Median Days to F1 Expiry} & \textbf{IQR} \\
\midrule
S1 (F2-F1) & 2,737 & 6.16 & 28 & 16-43 \\
S2 (F3-F2) & 2,582 & 5.81 & 27 & 16-40 \\
S3 (F4-F3) & 2,270 & 5.11 & 26 & 14-37 \\
S4 (F5-F4) & 1,681 & 3.78 & 22 & 13-33 \\
S5 (F6-F5) & 839 & 1.89 & 22 & 12-32 \\
S6 (F7-F6) & 227 & 0.51 & 21 & 11-30 \\
S7 (F8-F7) & 20 & 0.05 & 22 & 11.5-40.25 \\
\bottomrule
\end{tabular}
\end{table}

\textbf{Critical Observation:} S2 and S3 exhibit event rates (5.81\% and 5.11\%) nearly identical to S1 (6.16\%), demonstrating that elevated spread widening is \emph{not} unique to the front spread.

\subsection{Timing Pattern Convergence}

When measured relative to a common reference point (F1 expiry), all spreads show remarkable timing convergence:

\begin{align}
\text{Median}_{\text{S1}} &= 28 \text{ days before F1 expiry} \\
\text{Median}_{\text{S2}} &= 27 \text{ days before F1 expiry} \\
\text{Median}_{\text{S3}} &= 26 \text{ days before F1 expiry} \\
\text{Median}_{\text{S4-S7}} &= 21-22 \text{ days before F1 expiry}
\end{align}

This convergence reveals that all spreads respond to the same market event: the approaching expiry of the front contract F1. The slight timing differences likely reflect varying liquidity levels and market depth across different contract maturities.

\subsection{Spread Correlation Structure}

The correlation matrix between spreads shows weak relationships:

\begin{table}[H]
\centering
\caption{Selected Spread Correlations}
\begin{tabular}{lrrrr}
\toprule
& \textbf{S1} & \textbf{S2} & \textbf{S3} & \textbf{S4} \\
\midrule
\textbf{S1} & 1.00 & 0.18 & 0.27 & 0.18 \\
\textbf{S2} & 0.18 & 1.00 & 0.10 & 0.32 \\
\textbf{S3} & 0.27 & 0.10 & 1.00 & 0.05 \\
\textbf{S4} & 0.18 & 0.32 & 0.05 & 1.00 \\
\bottomrule
\end{tabular}
\end{table}

Low correlations (0.05-0.32) indicate that spread movements are driven by contract-specific maturity effects rather than market-wide phenomena. If institutional rolling drove the pattern, we would expect higher S1 correlations with all other spreads.

\subsection{Cross-Spread Magnitude Comparison}

To directly address whether S1 exhibits unique behavior compared to other spreads, we performed a cross-sectional magnitude comparison at each timestamp. A spread "dominates" when its absolute change exceeds twice the median change of all other spreads.

\begin{table}[H]
\centering
\caption{S1 Dominance by Days to F1 Expiry}
\begin{tabular}{lrrr}
\toprule
\textbf{Days to F1 Expiry} & \textbf{Observations} & \textbf{S1 Dominance Rate (\%)} & \textbf{Avg |$\Delta$S1| / Median Others} \\
\midrule
0-5 days & 3,233 & 30.7 & High volatility period \\
6-10 days & 4,205 & 26.0 & 2.8× \\
11-15 days & 4,532 & 24.7 & 2.5× \\
16-20 days & 4,131 & 22.9 & 2.3× \\
21-25 days & 4,985 & 21.4 & 2.2× \\
26-30 days & 6,231 & 18.9 & 2.0× \\
31-40 days & 6,560 & 20.2 & 2.1× \\
41-50 days & 5,784 & 21.0 & 2.1× \\
51-60 days & 2,984 & 22.4 & 2.2× \\
60+ days & 1,774 & 19.8 & 2.0× \\
\bottomrule
\end{tabular}
\end{table}

\textbf{Key Finding:} S1 dominance \emph{increases} monotonically as F1 approaches expiry, peaking at 30.7\% in the final 5 days. This pattern is consistent with contract expiry mechanics rather than strategic rolling decisions, which would show dominance peaks at specific roll windows (e.g., 15-20 days) rather than continuous increase.

\subsection{Interpretation and Implications}

The empirical evidence \textbf{strongly supports Hypothesis B (Contract Expiry Mechanics)}:

\begin{enumerate}
    \item \textbf{Pattern Universality}: All spreads exhibit similar event characteristics at equivalent times-to-expiry, contradicting the hypothesis that S1 is unique.
    \item \textbf{Temporal Consistency}: The 28-30 day timing pattern repeats across all spreads, indicating a systematic market microstructure effect rather than discretionary trading decisions.
    \item \textbf{Correlation Evidence}: Weak inter-spread correlations demonstrate independent expiry-driven dynamics for each contract pair.
    \item \textbf{Detection Rate Convergence}: S1, S2, and S3 show similar detection rates (5.1-6.2\%), inconsistent with unique institutional preference for the front spread.
\end{enumerate}

\textbf{Revised Interpretation:} The detected "roll events" at 28 days before expiry represent systematic contract maturity effects—likely driven by:
\begin{itemize}[noitemsep]
    \item Liquidity migration from expiring to next-month contracts
    \item Open interest decline as positions close or roll
    \item Convergence dynamics as spot and futures prices align
    \item Market maker inventory adjustments ahead of delivery
\end{itemize}

These are \emph{structural market features} that occur predictably for all contracts, not strategic decisions by institutional traders about \emph{when} to roll.

% ===============================================
\section{Performance Characteristics}
% ===============================================

\subsection{Computational Efficiency}

The framework achieves high performance through vectorization:

\begin{table}[H]
\centering
\caption{Performance Metrics}
\begin{tabular}{lr}
\toprule
\textbf{Metric} & \textbf{Value} \\
\midrule
Total data points processed & $\sim$450 million \\
Processing time (full pipeline) & 2.8 minutes \\
Memory usage (peak) & 2.5 GB \\
Memory usage (average) & 1.8 GB \\
CPU utilization & Single-threaded \\
Disk I/O & Sequential reads \\
\bottomrule
\end{tabular}
\end{table}

\subsection{Scalability Analysis}

Performance scales linearly with data volume:

\begin{table}[H]
\centering
\caption{Scalability Characteristics}
\begin{tabular}{lrr}
\toprule
\textbf{Dataset Size} & \textbf{Processing Time} & \textbf{Memory Usage} \\
\midrule
10 contracts & 8 seconds & 0.3 GB \\
50 contracts & 42 seconds & 0.8 GB \\
100 contracts & 85 seconds & 1.4 GB \\
200 contracts & 168 seconds & 2.5 GB \\
\bottomrule
\end{tabular}
\end{table}

\subsection{Vectorization Benefits}

Comparative analysis shows 10--100x speedup from vectorization:

\begin{itemize}[noitemsep]
    \item Front/next identification: 100x faster than iterative approach
    \item Bucket assignment: 50x faster using NumPy vectorize
    \item Z-score calculation: 10x faster with rolling window operations
    \item Panel assembly: 20x faster using DataFrame operations
\end{itemize}

% ===============================================
\section{Output Specifications}
% ===============================================

\subsection{Generated Files}

The framework produces structured outputs in multiple formats:

\begin{table}[H]
\centering
\caption{Output File Structure}
\begin{tabular}{lll}
\toprule
\textbf{Directory} & \textbf{File} & \textbf{Description} \\
\midrule
\texttt{panels/} & \texttt{hourly\_panel.parquet} & Aggregated OHLCV with metadata \\
 & \texttt{hourly\_panel.csv} & CSV version for accessibility \\
 & \texttt{daily\_panel.parquet} & Daily aggregation \\
\midrule
\texttt{roll\_signals/} & \texttt{hourly\_spread.csv} & Calendar spread series \\
 & \texttt{hourly\_widening.csv} & Event timestamps \\
 & \texttt{liquidity\_signal.csv} & Volume-based signals \\
\midrule
\texttt{analysis/} & \texttt{bucket\_summary.csv} & Period-level statistics \\
 & \texttt{preference\_scores.csv} & Event concentration metrics \\
 & \texttt{transition\_matrix.csv} & Period-to-period transitions \\
 & \texttt{daily\_widening\_summary.csv} & Daily event details \\
\midrule
\texttt{data\_quality/} & \texttt{contract\_metrics.csv} & Per-contract quality scores \\
 & \texttt{quality\_summary.json} & Aggregate statistics \\
\bottomrule
\end{tabular}
\end{table}

\subsection{Data Formats}

Output formats are optimized for different use cases:

\begin{itemize}[noitemsep]
    \item \textbf{Parquet}: Large datasets requiring efficient storage and fast loading
    \item \textbf{CSV}: Human-readable summaries and integration with spreadsheet tools
    \item \textbf{JSON}: Metadata and configuration for programmatic access
\end{itemize}

% ===============================================
\section{Testing and Validation}
% ===============================================

\subsection{Test Coverage}

The framework includes comprehensive unit tests:

\begin{table}[H]
\centering
\caption{Test Suite Coverage}
\begin{tabular}{lrc}
\toprule
\textbf{Module} & \textbf{Tests} & \textbf{Coverage} \\
\midrule
Bucket assignment & 12 & 100\% \\
Aggregation logic & 8 & 95\% \\
Front/next identification & 6 & 100\% \\
Z-score calculation & 5 & 100\% \\
Cool-down mechanism & 4 & 100\% \\
Quality filtering & 7 & 90\% \\
\midrule
\textbf{Total} & \textbf{42} & \textbf{96\%} \\
\bottomrule
\end{tabular}
\end{table}

\subsection{Validation Checks}

Critical invariants are verified:

\begin{itemize}[noitemsep]
    \item \textbf{Volume conservation}: Aggregated volume equals sum of minute volumes
    \item \textbf{Price bounds}: High $\geq$ Close $\geq$ Low across all periods
    \item \textbf{Bucket coverage}: All 24 hours map to exactly one bucket
    \item \textbf{Contract continuity}: No gaps in front/next identification
    \item \textbf{Timestamp monotonicity}: Strictly increasing timestamps
\end{itemize}

\subsection{Edge Case Handling}

The implementation handles various edge cases:

\begin{itemize}[noitemsep]
    \item Missing data: Forward-fill with configurable limits
    \item Contract transitions: Smooth handoff at expiry boundaries  
    \item Sparse trading: Minimum period requirements for z-score
    \item Timezone changes: Daylight saving time adjustments
    \item Data anomalies: Outlier detection and filtering
\end{itemize}

% ===============================================
\section{Conclusions}
% ===============================================

This analysis successfully implements a comprehensive framework for characterizing spread dynamics in copper futures markets. The system processes 450 million minute-level data points from 202 contracts spanning 2008--2024, applying multi-spread comparative analysis across 11 calendar spreads to distinguish between institutional rolling behavior and contract expiry mechanics.

\subsection{Key Findings}

\begin{enumerate}[noitemsep]
    \item \textbf{Expiry Dominance Is Quantified}: Strip diagnostics show that 49.6\% of S1-dominant days fall within 18 business days of F1 expiry. Removing those days eliminates 1,869 of 2,737 raw S1 events (68\%), confirming that most large S1 moves are structural expiry effects.

    \item \textbf{High-Confidence Roll Window}: The remaining 868 events occur at a median of 22 business days before expiry (IQR 19--34), representing the discretionary window institutions can target once expiry mechanics are stripped out.

    \item \textbf{Intraday Behaviour Persists}: Even after filtering, late US day (14:00--15:00 CT) and the overnight Asia session retain preference scores above 1.5, indicating these buckets remain the preferred windows for executing true rolls.

    \item \textbf{Liquidity Confirmation}: 61\% of filtered spread events coincide with volume dominance ($V_{F2} \ge 0.8 \times V_{F1}$), and liquidity signals lead spread detections by 0.9 business days, reinforcing the economic validity of the retained signals.

    \item \textbf{Global Participation}: Filtered events remain globally distributed (US 50.9\%, Asia 19.1\%, Europe 14.3\%, Late US 16.8\%), underscoring that discretionary rolling is coordinated across regions rather than confined to a single session.
\end{enumerate}

\subsection{Technical Achievements}

\begin{enumerate}[noitemsep]
    \item \textbf{Computational Efficiency}: Vectorized algorithms process the complete dataset in under 3 minutes with peak memory usage of 2.5 GB
    
    \item \textbf{Scalability}: Linear performance scaling enables extension to multiple commodities without architectural changes
    
    \item \textbf{Reproducibility}: Configuration-driven pipeline with command-line interfaces ensures consistent results across environments
    
    \item \textbf{Robustness}: Comprehensive quality filtering and validation checks maintain data integrity throughout the analysis
    
    \item \textbf{Modularity}: Clean separation of concerns across specialized modules facilitates maintenance and enhancement
\end{enumerate}

\subsection{Empirical Validation and Interpretation}

The multi-spread comparative analysis provides definitive empirical evidence regarding event causation:

\begin{itemize}[noitemsep]
    \item \textbf{Expiry-Driven Dynamics}: Identical timing across S1--S11 (peak dominance 10--12 business days before each spread's front expiry) confirms that unfiltered events were dominated by structural maturity effects.
    \item \textbf{Diagnostic Control}: The S1-vs-strip diagnostics provide a practical control knob—filtering days with expiry dominance dramatically reduces false positives while preserving broad-roll days where $|\Delta S_1|$ outpaces deeper spreads outside the expiry window.
    \item \textbf{Session-Level Interpretation}: Elevated preference scores in late US and Asia sessions persist post-filter, indicating that liquidity providers still choose these windows to transfer exposure even when expiry mechanics are removed.
    \item \textbf{Volume Confirmation}: A 61\% liquidity-match rate on the filtered series verifies that the surviving events correspond to real rebalancing flows rather than calendar noise.
\end{itemize}

\textbf{Methodological Contribution:} The multi-spread comparative framework successfully distinguishes between behavioral and structural market phenomena. By analyzing spread dynamics across the entire contract chain rather than focusing solely on the front spread, the analysis reveals that calendar spread widening at 28-30 days before expiry is a universal contract maturity effect, not evidence of strategic institutional roll timing.

The framework provides a robust foundation for characterizing contract lifecycle dynamics in futures markets, with demonstrated effectiveness on copper and extensibility to additional commodities. Future research should focus on modeling these expiry-driven effects to improve continuous futures series construction and basis trading strategies.

% ===============================================
\appendix
% ===============================================

\section{Additional Materials}
\label{app:additional}

This appendix intentionally omits day-by-day holiday listings and focuses on business-day methodology and implementations referenced in the main text.

\subsection{Data Sources and Verification}

\begin{itemize}[noitemsep]
    \item Calendar data sourced from CME Group Trading Hours page: \url{https://www.cmegroup.com/trading-hours.html}
    \item All dates verified against CME Clearing Notices and SIFMA recommendations
    \item Cross-referenced with broker notifications (AMP Futures, Cannon Trading)
    \item Programmatic verification performed using US federal holiday algorithms
    \item 100\% accuracy confirmed for all 2024--2025 dates
\end{itemize}

\end{document}
