\documentclass[11pt]{article}
\usepackage[margin=1in]{geometry}
\usepackage{graphicx}
\usepackage{booktabs}
\usepackage{hyperref}
\usepackage{longtable}
\usepackage{amsmath}
\usepackage{amssymb}
\usepackage{float}

\title{Intra-Month Seasonality in Copper Calendar Spreads\\ \large Empirical Evidence from CME Futures (2015--2024)}
\author{Automated Analytics Pipeline}
\date{\today}

\begin{document}
\maketitle

\section{Introduction}

This report investigates systematic behavior in the CME Copper (HG) calendar spread market, specifically focusing on the behavior of the front spread (S1, defined as F2$-$F1) during the final five trading days of each month (EOM window).

Institutional roll activity---the simultaneous liquidation of expiring positions (F1) and establishment of new positions (F2)---is hypothesized to create predictable demand imbalances. This study quantifies these effects using high-frequency data aggregated to daily session means over a ten-year period (2015--2024).

The analysis aims to answer three primary questions:
\begin{enumerate}
    \item Does the S1 spread exhibit a systematic widening trend during the month-end window?
    \item Is this pattern strengthening over time?
    \item Can the effect be attributed to spread-specific dynamics rather than outright price trends?
\end{enumerate}

\section{Data and Methodology}

\subsection{Dataset}
The study utilizes minute-level OHLCV data for CME High Grade Copper futures from January 2015 through December 2024 (119 months). Analysis is restricted to the US regular trading session (09:00--15:00 CT) to ensure liquidity and exclude overnight volatility. The final dataset consists of daily volume-weighted average prices (VWAP) for the front (F1) and second (F2) contracts.

\subsection{Variable Definitions}
\begin{itemize}
    \item \textbf{S1 (Spread)}: $P_{F2} - P_{F1}$ in USD.
    \item \textbf{F1 (Price Level)}: $P_{F1}$ in USD (control variable).
    \item \textbf{S1\_ANOM (Anomaly)}: $S1_t - \overline{S1}_{month}$, removing the monthly mean level to isolate intra-month shape.
\end{itemize}

\subsection{Regression Framework}
The analysis employs a two-stage regression approach:

\textbf{Stage 1: Intra-Month Slope Estimation.} For each month $m$, we estimate the daily slope ($\beta_m$) of the variable of interest over the last five trading days (EOM$-4$ to EOM):
\begin{equation}
Y_{t} = \alpha_m + \beta_m \cdot k_t + \varepsilon_{t}
\end{equation}
where $k_t \in \{1, \dots, 5\}$ represents the trading day index within the EOM window.

\textbf{Stage 2: Time Trend Analysis.} We regress the resulting monthly slopes ($\beta_m$) against time ($t_m$, month index since Jan 2015) to detect structural changes:
\begin{equation}
\beta_m = \gamma_0 + \gamma_1 \cdot t_m + \eta_m
\end{equation}
A statistically significant positive $\gamma_1$ indicates the intra-month widening is accelerating over time.

\section{Empirical Results}

\subsection{Average Intra-Month Behavior}
Table \ref{tab:slope_summary} summarizes the average intra-month slopes over the full sample period.

\begin{table}[H]
\centering
\caption{Average Intra-Month Slope Statistics (2015--2024)}
\begin{tabular}{lrrrr}
\toprule
\textbf{Series} & \textbf{Mean Slope} & \textbf{Median Slope} & \textbf{\% Positive} & \textbf{$p$-value ($H_0 \le 0$)} \\
 & (cents/day) & (cents/day) & & \\
\midrule
S1 & \textbf{+0.212} & +0.149 & 75.6\% & $<$0.001 \\
F1 & $-$0.198 & $-$0.176 & 46.2\% & 0.271 \\
S1\_ANOM & \textbf{+0.212} & +0.149 & 75.6\% & $<$0.001 \\
\bottomrule
\end{tabular}
\label{tab:slope_summary}
\end{table}

The S1 spread shows a statistically significant average widening of 0.212 cents per day during the EOM window. Over the 5-day period, this accumulates to approximately 1.06 cents. The F1 price level shows no significant trend ($p=0.271$), confirming that the effect is specific to the spread relationship and not a byproduct of a general price trend.

\subsection{Structural Change Over Time}
Table \ref{tab:time_trend} presents the results of the time trend regression (Stage 2), assessing whether the slope $\beta_m$ has changed over the decade.

\begin{table}[H]
\centering
\caption{Time Trend in Monthly Slopes ($\beta_m = \gamma_0 + \gamma_1 \cdot t_m$)}
\begin{tabular}{lrrrr}
\toprule
\textbf{Series} & \textbf{2015 Level} & \textbf{Annual Drift ($\gamma_1 \times 12$)} & \textbf{$R^2$} & \textbf{$p$-value} \\
 & (cents/day) & (cents/year) & & \\
\midrule
S1 & 0.201 & \textbf{+0.039} & 0.117 & $<$0.001 \\
F1 & $-$0.198 & $-$0.068 & 0.010 & 0.280 \\
S1\_ANOM & 0.201 & \textbf{+0.039} & 0.117 & $<$0.001 \\
\bottomrule
\end{tabular}
\label{tab:time_trend}
\end{table}

The S1 slope has increased by approximately 0.039 cents per year. Over the 10-year sample, the daily widening rate has grown from $\sim$0.20 cents/day in 2015 to $\sim$0.59 cents/day in 2024, representing an approximate 40\% increase in the intensity of the effect.

\subsection{Timing Decomposition}
To pinpoint the timing of this intensification, we analyze level trends for specific days within the window (Table \ref{tab:offset_trends}).

\begin{table}[H]
\centering
\caption{Level Trends by EOM Offset (S1 Series)}
\begin{tabular}{lrrrr}
\toprule
\textbf{Day} & \textbf{Trend} (cents/year) & \textbf{$p$-value} & \textbf{$R^2$} & \textbf{Status} \\
\midrule
EOM$-4$ & $-$0.075 & 0.075 & 0.029 & No Trend \\
EOM$-3$ & +0.013 & 0.721 & 0.001 & No Trend \\
EOM$-2$ & +0.020 & 0.602 & 0.003 & No Trend \\
EOM$-1$ & \textbf{+0.079} & \textbf{0.005} & 0.066 & \textbf{Strengthening} \\
EOM & \textbf{+0.074} & \textbf{0.016} & 0.048 & \textbf{Strengthening} \\
\bottomrule
\end{tabular}
\label{tab:offset_trends}
\end{table}

The structural strengthening is concentrated in the final two trading days (EOM$-1$ and EOM), suggesting that roll activity has become increasingly compressed at the very end of the month.

\section{Discussion}

\subsection{Mechanisms}
The observed widening is consistent with ``roll pressure'' hypotheses. As long-only commodity indices and ETFs roll their positions forward (selling F1, buying F2), they exert widening pressure on the calendar spread. The persistence of this pattern suggests that the liquidity premium for absorbing these flows has not been arbitraged away, and may in fact be increasing due to capital constraints on market makers or the growth of passive investment vehicles.

\subsection{Control Analysis}
The absence of a significant trend in the F1 price slope (Table \ref{tab:slope_summary}) is a crucial validation. It implies that the S1 widening is not simply a reflection of the front contract price crashing (e.g., due to delivery concerns) but rather a relative repricing of the term structure.

\section{Implications}

\subsection{Execution Costs}
For institutional investors rolling positions, the EOM window represents a high-cost execution period. Based on 2024 estimates, rolling a standard position across the 5-day window incurs an expected spread penalty of approximately 1.0--1.2 cents per contract. This cost has risen significantly relative to 2015 levels.

\subsection{Recommendations}
\begin{itemize}
    \item \textbf{Early Roll Execution}: Shifting roll activity to the EOM$-10$ to EOM$-6$ window avoids the systematic widening period.
    \item \textbf{Spread Trading}: The systematic nature of the widening presents a potential risk premium harvesting strategy (long spread early month, exit before EOM), subject to transaction cost analysis.
    \item \textbf{Monitoring}: Continued monitoring of the EOM$-1$ and EOM trends is recommended to detect further intensification or potential reversal.
\end{itemize}

\newpage
\section{Appendix: Visualizations}

\begin{figure}[H]
\centering
\includegraphics[width=0.48\textwidth]{../outputs/seasonality/s1_level/s1_overlay_2024.png}
\includegraphics[width=0.48\textwidth]{../outputs/seasonality/s1_level/s1_average_2024.png}
\caption{S1 Level by Business Day of Month (2024). Left: Individual months. Right: Mean with IQR. Note the upward drift in the final days (right side of x-axis).}
\end{figure}

\begin{figure}[H]
\centering
\includegraphics[width=0.48\textwidth]{../outputs/seasonality/s1_delta/s1_overlay_2024.png}
\includegraphics[width=0.48\textwidth]{../outputs/seasonality/s1_delta/s1_average_2024.png}
\caption{S1 Daily Change by Business Day of Month (2024). Positive deltas cluster at month-end.}
\end{figure}

\end{document}
