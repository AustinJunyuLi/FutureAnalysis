\documentclass[11pt]{article}
\usepackage[margin=1in]{geometry}
\usepackage{graphicx}
\usepackage{booktabs}
\usepackage{hyperref}
\usepackage{longtable}
\usepackage{amsmath}
\usepackage{amssymb}
\usepackage{float}

\title{Intra-Month Seasonality in Copper Calendar Spreads\\ \large Empirical Evidence from CME Futures (2015--2024)}
\author{Automated Analytics Pipeline}
\date{\today}

\begin{document}
\maketitle

\section{Introduction}

This report investigates systematic behavior in the CME Copper (HG) calendar spread market, specifically focusing on the behavior of the front spread (S1, defined as F2$-$F1) during the final five trading days of each month (EOM window).

Institutional roll activity---the simultaneous liquidation of expiring positions (F1) and establishment of new positions (F2)---is hypothesized to create predictable demand imbalances. This study quantifies these effects using high-frequency data aggregated to daily session means over a ten-year period (2015--2024).

The analysis aims to answer three primary questions:
\begin{enumerate}
    \item Does the S1 spread exhibit a systematic widening trend during the month-end window?
    \item Is this pattern strengthening over time?
    \item Can the effect be attributed to spread-specific dynamics rather than outright price trends?
\end{enumerate}

\section{Data and Methodology}

\subsection{Dataset}
The study utilizes minute-level OHLCV data for CME High Grade Copper futures from January 2015 through December 2024 (119 months). Analysis is restricted to the US regular trading session (09:00--15:00 CT) to ensure liquidity and exclude overnight volatility. The final dataset consists of daily volume-weighted average prices (VWAP) for the front (F1) and second (F2) contracts.

\subsection{Data Construction}
Contract identification follows a deterministic expiry-based approach using CME official expiry timestamps. For each trading day, F1 is defined as the contract with the nearest expiry date, F2 as the second nearest, and so forth. This ensures unambiguous strip assignment and aligns with how institutional traders view the term structure.

Daily VWAP is computed as the volume-weighted average across seven intraday buckets spanning 09:00--15:00 CT (CME Globex regular session hours). This restriction excludes overnight and early-morning periods where liquidity is thin and price discovery may be less reliable. Holidays follow the CME Globex calendar; days with partial or early closures are excluded from the sample.

The end-of-month (EOM) window is defined using \textit{business days of month} (BDOM) rather than calendar dates. EOM denotes the last trading day of the calendar month, EOM$-1$ the penultimate trading day, and so on. This normalization accounts for varying month lengths and weekend/holiday patterns.

\subsection{Variable Definitions}
\begin{itemize}
    \item \textbf{S1 (Spread)}: $P_{F2} - P_{F1}$ in USD.
    \item \textbf{F1 (Price Level)}: $P_{F1}$ in USD (control variable).
    \item \textbf{S1\_ANOM (Anomaly)}: $S1_t - \overline{S1}_{month}$, removing the monthly mean level to isolate intra-month shape.
\end{itemize}

\subsection{Regression Framework}
The analysis employs a two-stage regression approach:

\textbf{Stage 1: Intra-Month Slope Estimation.} For each month $m$, we estimate the daily slope ($\beta_m$) of the variable of interest over the last five trading days (EOM$-4$ to EOM):
\begin{equation}
Y_{t} = \alpha_m + \beta_m \cdot k_t + \varepsilon_{t}
\end{equation}
where $k_t \in \{1, \dots, 5\}$ represents the trading day index within the EOM window.

\textbf{Stage 2: Time Trend Analysis.} We regress the resulting monthly slopes ($\beta_m$) against time ($t_m$, month index since Jan 2015) to detect structural changes:
\begin{equation}
\beta_m = \gamma_0 + \gamma_1 \cdot t_m + \eta_m
\end{equation}
A statistically significant positive $\gamma_1$ indicates the intra-month widening is accelerating over time.

\section{Empirical Results}

\subsection{Average Intra-Month Behavior}
Table \ref{tab:slope_summary} summarizes the average intra-month slopes over the full sample period.

\begin{table}[H]
\centering
\caption{Average Intra-Month Slope Statistics (2015--2024)}
\begin{tabular}{lrrrr}
\toprule
\textbf{Series} & \textbf{Mean Slope} & \textbf{Median Slope} & \textbf{\% Positive} & \textbf{$p$-value ($H_0 \le 0$)} \\
 & (cents/day) & (cents/day) & & \\
\midrule
S1 & \textbf{+0.212} & +0.149 & 75.6\% & $<$0.001 \\
F1 & $-$0.198 & $-$0.176 & 46.2\% & 0.271 \\
S1\_ANOM & \textbf{+0.212} & +0.149 & 75.6\% & $<$0.001 \\
\bottomrule
\end{tabular}
\label{tab:slope_summary}
\end{table}

The S1 spread shows a statistically significant average widening of 0.212 cents per day during the EOM window. Over the 5-day period, this accumulates to approximately 1.06 cents. Visual confirmation of this pattern is provided in Appendix Figure A.1, which shows a clear upward drift in S1 levels during the BDOM 15--20 window (final trading days). Figure A.2 further illustrates that positive daily changes (deltas) systematically cluster at month-end, confirming the widening acceleration.

The F1 price level shows no significant trend ($p=0.271$), confirming that the effect is specific to the spread relationship and not a byproduct of a general price trend.

\subsection{Structural Change Over Time}
Table \ref{tab:time_trend} presents the results of the time trend regression (Stage 2), assessing whether the slope $\beta_m$ has changed over the decade.

\begin{table}[H]
\centering
\caption{Time Trend in Monthly Slopes ($\beta_m = \gamma_0 + \gamma_1 \cdot t_m$)}
\begin{tabular}{lrrrr}
\toprule
\textbf{Series} & \textbf{2015 Level} & \textbf{Annual Drift ($\gamma_1 \times 12$)} & \textbf{$R^2$} & \textbf{$p$-value} \\
 & (cents/day) & (cents/year) & & \\
\midrule
S1 & 0.201 & \textbf{+0.039} & 0.117 & $<$0.001 \\
F1 & $-$0.198 & $-$0.068 & 0.010 & 0.280 \\
S1\_ANOM & 0.201 & \textbf{+0.039} & 0.117 & $<$0.001 \\
\bottomrule
\end{tabular}
\label{tab:time_trend}
\end{table}

The S1 slope has increased by approximately 0.039 cents per year. Over the 10-year sample, the daily widening rate has grown from $\sim$0.20 cents/day in 2015 to $\sim$0.59 cents/day in 2024, representing an approximate 40\% increase in the intensity of the effect.

\subsection{Timing Decomposition}
To pinpoint the timing of this intensification, we analyze level trends for specific days within the window (Table \ref{tab:offset_trends}).

\begin{table}[H]
\centering
\caption{Level Trends by EOM Offset (S1 Series)}
\begin{tabular}{lrrrr}
\toprule
\textbf{Day} & \textbf{Trend} (cents/year) & \textbf{$p$-value} & \textbf{$R^2$} & \textbf{Status} \\
\midrule
EOM$-4$ & $-$0.075 & 0.075 & 0.029 & No Trend \\
EOM$-3$ & +0.013 & 0.721 & 0.001 & No Trend \\
EOM$-2$ & +0.020 & 0.602 & 0.003 & No Trend \\
EOM$-1$ & \textbf{+0.079} & \textbf{0.005} & 0.066 & \textbf{Strengthening} \\
EOM & \textbf{+0.074} & \textbf{0.016} & 0.048 & \textbf{Strengthening} \\
\bottomrule
\end{tabular}
\label{tab:offset_trends}
\end{table}

The structural strengthening is concentrated in the final two trading days (EOM$-1$ and EOM), suggesting that roll activity has become increasingly compressed at the very end of the month.

\subsection{Back Spread Comparison: S2 Analysis}

To assess whether the EOM widening pattern extends beyond the front spread, we examine the S2 spread (F3$-$F2) using the same business-day-of-month methodology. Visual inspection of S2 level and delta patterns (Appendix Figures A.3--A.4) reveals markedly weaker and less consistent EOM behavior compared to S1.

Unlike S1, which exhibits a clear upward drift in the BDOM 15--20 window across most months, S2 shows substantial inter-month variation with no systematic directional pattern. This finding is consistent with the hypothesis that roll pressure is concentrated in the front spread, where institutional flows are largest and liquidity transitions most pronounced. The absence of a comparable S2 effect strengthens the interpretation that S1 widening is driven by front-contract roll dynamics rather than a general term structure phenomenon.

Quantitative regression analysis for S2 (analogous to Tables \ref{tab:slope_summary}--\ref{tab:offset_trends}) is left to future work, but the visual evidence supports the specificity of the S1 result.

\section{Discussion}

\subsection{Mechanisms}

The systematic EOM widening documented above can be attributed to several interconnected mechanisms:

\textbf{Roll Pressure.} Large passive investors---commodity index funds, ETFs, and benchmark-tracking portfolios---must periodically roll their positions from expiring contracts (F1) to the next available contract (F2) to maintain continuous exposure. This rolling activity is mechanically imbalanced: it involves simultaneous selling of F1 and buying of F2, which widens the calendar spread (S1 = F2 $-$ F1). While roll schedules are publicly disclosed, the sheer size of these flows creates predictable demand imbalances.

Major commodity indices (e.g., S\&P GSCI, Bloomberg Commodity Index) typically roll during the 5th--9th business days of the month, not at month-end. However, not all institutional investors follow index schedules precisely. Corporate hedgers, non-index CTAs, and discretionary funds may delay rolling until month-end for operational or accounting reasons, creating residual EOM pressure even after index rolls complete.

\textbf{Liquidity Fragmentation.} As F1 approaches expiry, open interest migrates to F2, fragmenting liquidity across the term structure. Market depth in F1 thins, while F2 becomes the new liquidity focal point. This transition amplifies the price impact of any directional flow (including rolls), widening bid-ask spreads and increasing the cost of closing F1 positions.

\textbf{Passive Capital Growth.} The documented strengthening of the EOM effect (Table \ref{tab:time_trend}, +40\% over 2015--2024) coincides with the secular growth of commodity index investing and passive vehicles. As passive assets under management increase, the magnitude of predictable roll flows grows, intensifying calendar spread distortions. This trend has been documented across multiple commodity markets (crude oil, natural gas, grains) and is not specific to copper.

\textbf{Arbitrage Limits.} While the EOM widening is systematic and predictable, it has not been fully arbitraged away. Market-making capital is constrained by balance sheet costs (post-crisis regulation), margin requirements, and basis risk. Arbitrageurs who attempt to fade the widening (short F2, long F1) face the risk that spreads widen further before mean-reverting, especially if roll flows are larger or more concentrated than anticipated. The persistence of the effect suggests that risk-adjusted returns to arbitrage are insufficient to attract enough capital to eliminate the pattern.

\subsection{Control Analysis}
The absence of a significant trend in the F1 price slope (Table \ref{tab:slope_summary}) is a crucial validation. It implies that the S1 widening is not simply a reflection of the front contract price crashing (e.g., due to delivery concerns) but rather a relative repricing of the term structure.

\subsection{Statistical Inference and Robustness}

All regression standard errors reported in this study use asymptotic normal approximations based on OLS residuals. While this approach is standard for preliminary analysis, several caveats warrant acknowledgment:

\textbf{Heteroskedasticity.} Calendar spread volatility may vary across months (e.g., heightened volatility during macroeconomic stress periods). Heteroskedasticity-robust (White/HC) standard errors would provide more conservative inference. Given the large sample size (119 months) and the magnitude of the $t$-statistics (e.g., $t = 7.04$ for S1 average slope), qualitative conclusions are unlikely to change under robust inference.

\textbf{Serial Correlation.} Monthly slope estimates ($\beta_m$) may exhibit autocorrelation if EOM patterns persist across adjacent months. Newey-West HAC standard errors or clustered inference (by year) would account for this structure. The current analysis does not adjust for clustering, which may slightly overstate statistical significance in the time trend regressions (Table \ref{tab:time_trend}).

\textbf{Small Sample at Offset Level.} The level regressions (Table \ref{tab:offset_trends}) use $n = 91$--$117$ observations per offset, which is adequate for asymptotic approximations but modest. Bootstrap confidence intervals would provide finite-sample validation, particularly for the marginally significant EOM$-4$ trend ($p = 0.075$).

Despite these caveats, the core finding---systematic S1 widening at EOM, strengthening over time---is robust across multiple specifications (raw S1, anomaly-adjusted S1\_ANOM) and is visually confirmed in independent 2024 data (Appendix Figures A.1--A.2). Formal robustness checks (bootstrap, HAC inference, out-of-sample validation) are recommended for publication-quality analysis but are beyond the scope of this initial exploratory report.

\section{Implications}

\subsection{Execution Costs}

For institutional investors rolling positions, the EOM window represents a high-cost execution period. Based on 2024 estimates, rolling a standard position across the 5-day window incurs an expected spread penalty of approximately 1.0--1.2 cents per contract.

\textbf{Economic Magnitude.} CME High Grade Copper futures have a contract size of 25,000 pounds. At a representative F1 price of \$4.00/lb (2024 average), one contract has a notional value of \$100,000. The 1.06-cent EOM widening therefore represents:
\begin{itemize}
    \item \textbf{Absolute cost}: 1.06 cents/lb $\times$ 25,000 lbs = \$265 per contract
    \item \textbf{Percentage impact}: 0.0265\% of notional value
    \item \textbf{Relative to bid-ask}: Typical quoted spread is 0.05--0.10 cents; EOM widening is 10--20$\times$ the bid-ask
    \item \textbf{Relative to daily volatility}: Average daily range in 2024 was $\sim$5--10 cents; EOM cumulative impact is 10--20\% of daily noise
\end{itemize}

While 26.5 basis points may appear modest, it represents a \textit{systematic} and \textit{predictable} cost. For a large institutional investor rolling 10,000 contracts monthly, the EOM penalty totals \$2.65 million annually. Furthermore, the strengthening documented in Table \ref{tab:time_trend} implies that this cost has increased by approximately 40\% from 2015 to 2024, raising the stakes for roll timing optimization.

\subsection{Recommendations}
\begin{itemize}
    \item \textbf{Early Roll Execution}: Shifting roll activity to the EOM$-10$ to EOM$-6$ window avoids the systematic widening period.
    \item \textbf{Spread Trading}: The systematic nature of the widening presents a potential risk premium harvesting strategy (long spread early month, exit before EOM), subject to transaction cost analysis.
    \item \textbf{Monitoring}: Continued monitoring of the EOM$-1$ and EOM trends is recommended to detect further intensification or potential reversal.
\end{itemize}


\newpage
\section*{Appendix}
\addcontentsline{toc}{section}{Appendix}

\subsection*{A. Visualizations}

\begin{figure}[H]
\centering
\includegraphics[width=0.48\textwidth]{../outputs/seasonality/s1_level/s1_overlay_2024.png}
\includegraphics[width=0.48\textwidth]{../outputs/seasonality/s1_level/s1_average_2024.png}
\caption{S1 Level Patterns by Business Day of Month (2024). \textit{Left:} Overlay of all 12 months showing individual trajectories. \textit{Right:} Average with interquartile range (IQR) bands. Note the systematic upward drift during BDOM 15--20 (final trading days), confirming the widening documented in Table \ref{tab:slope_summary}.}
\label{fig:a1_s1_level}
\end{figure}

\begin{figure}[H]
\centering
\includegraphics[width=0.48\textwidth]{../outputs/seasonality/s1_delta/s1_overlay_2024.png}
\includegraphics[width=0.48\textwidth]{../outputs/seasonality/s1_delta/s1_average_2024.png}
\caption{S1 Daily Changes (Delta) by BDOM (2024). \textit{Left:} Overlay showing intra-month velocity across individual months. \textit{Right:} Average delta with IQR bands. Positive daily changes systematically cluster near month-end (BDOM 18--20), confirming acceleration of the widening pattern.}
\label{fig:a2_s1_delta}
\end{figure}

\begin{figure}[H]
\centering
\includegraphics[width=0.48\textwidth]{../outputs/seasonality/s2_level/s2_overlay_2024.png}
\includegraphics[width=0.48\textwidth]{../outputs/seasonality/s2_level/s2_average_2024.png}
\caption{S2 (F3$-$F2) Level Patterns by BDOM (2024). Unlike S1, the S2 back spread shows no consistent EOM widening pattern. Inter-month variation is high, and the average (right panel) exhibits no systematic directional drift in the final trading days. This validates that the EOM effect is specific to the front spread (S1), consistent with roll pressure concentration in F1/F2.}
\label{fig:a3_s2_level}
\end{figure}

\begin{figure}[H]
\centering
\includegraphics[width=0.48\textwidth]{../outputs/seasonality/s2_delta/s2_overlay_2024.png}
\includegraphics[width=0.48\textwidth]{../outputs/seasonality/s2_delta/s2_average_2024.png}
\caption{S2 Daily Changes (Delta) by BDOM (2024). S2 deltas show no systematic clustering at month-end, contrasting sharply with S1 (Figure A.2). The absence of a comparable S2 effect reinforces the interpretation that the documented widening is driven by front-contract roll dynamics rather than a general term structure phenomenon.}
\label{fig:a4_s2_delta}
\end{figure}

\subsection*{B. Full Regression Output Tables}

The tables below provide complete regression results underlying the summary statistics reported in the main text. All standard errors use OLS asymptotic normal approximations (see Section 4.3 for robustness caveats).

\begin{table}[H]
\centering
\caption{EOM Level Regressions: $Y_{t,k} = \alpha_k + \beta_k \cdot t_m + \varepsilon$ (S1, F1, S1\_ANOM)}
\begin{tabular}{llrrrrr}
\toprule
\textbf{Series} & \textbf{Offset} & \textbf{$\beta$/month} & \textbf{$\beta$/year} & \textbf{SE($\beta$)} & \textbf{$p$-value} & \textbf{$n$} \\
\midrule
S1 & EOM$-4$ & $-6.22 \times 10^{-5}$ & $-0.075$ & $3.49 \times 10^{-5}$ & 0.075 & 109 \\
S1 & EOM$-3$ & $+1.04 \times 10^{-5}$ & $+0.012$ & $2.92 \times 10^{-5}$ & 0.721 & 115 \\
S1 & EOM$-2$ & $+1.70 \times 10^{-5}$ & $+0.020$ & $3.27 \times 10^{-5}$ & 0.602 & 91 \\
S1 & EOM$-1$ & $+6.61 \times 10^{-5}$ & $+0.079$ & $2.36 \times 10^{-5}$ & 0.005 & 112 \\
S1 & EOM & $+6.14 \times 10^{-5}$ & $+0.074$ & $2.55 \times 10^{-5}$ & 0.016 & 117 \\
\midrule
F1 & EOM$-4$ & $+0.0183$ & $+0.220$ & 0.0012 & $<$0.001 & 109 \\
F1 & EOM$-3$ & $+0.0185$ & $+0.222$ & 0.0012 & $<$0.001 & 115 \\
F1 & EOM$-2$ & $+0.0180$ & $+0.216$ & 0.0012 & $<$0.001 & 91 \\
F1 & EOM$-1$ & $+0.0184$ & $+0.221$ & 0.0012 & $<$0.001 & 112 \\
F1 & EOM & $+0.0184$ & $+0.221$ & 0.0012 & $<$0.001 & 117 \\
\midrule
S1\_ANOM & EOM$-4$ & $-7.86 \times 10^{-5}$ & $-0.094$ & $2.62 \times 10^{-5}$ & 0.003 & 109 \\
S1\_ANOM & EOM$-3$ & $-6.37 \times 10^{-6}$ & $-0.008$ & $2.12 \times 10^{-5}$ & 0.764 & 115 \\
S1\_ANOM & EOM$-2$ & $-1.28 \times 10^{-5}$ & $-0.015$ & $2.75 \times 10^{-5}$ & 0.641 & 91 \\
S1\_ANOM & EOM$-1$ & $+6.02 \times 10^{-5}$ & $+0.072$ & $1.82 \times 10^{-5}$ & 0.001 & 112 \\
S1\_ANOM & EOM & $+5.69 \times 10^{-5}$ & $+0.068$ & $1.75 \times 10^{-5}$ & 0.001 & 117 \\
\bottomrule
\end{tabular}
\label{tab:full_level_regressions}
\end{table}

\textbf{Interpretation:} F1 shows a strong positive time trend at all offsets (reflecting secular price appreciation 2015--2024), but this trend is uniform across the EOM window---no offset-specific strengthening. In contrast, S1 and S1\_ANOM show significant strengthening \textit{only} at EOM$-1$ and EOM, consistent with increasingly concentrated roll pressure in the final two trading days.

\subsection*{C. Reproducibility and Data Access}

\textbf{Code Repository.} All analysis scripts are available in the project directory:
\begin{itemize}
    \item \texttt{scripts/s1\_eom\_trend.py} -- Constructs EOM data and computes level regressions
    \item \texttt{scripts/s1\_eom\_intra\_month\_slope.py} -- Computes monthly slopes and time trend analysis
    \item \texttt{scripts/s1\_overlay.py}, \texttt{scripts/s1\_average\_by\_bdom.py} -- Generates BDOM visualizations
\end{itemize}

\textbf{Replication Command.} To reproduce the full analysis pipeline:
\begin{verbatim}
python scripts/s1_eom_trend.py --min-eom-obs 3 --outdir outputs/s1_eom_trend_2015_2024
python scripts/s1_eom_intra_month_slope.py \
    --input outputs/s1_eom_trend_2015_2024/eom_data_long.csv \
    --outdir outputs/s1_eom_trend_2015_2024
\end{verbatim}

\textbf{Output Files.} Key results are stored in \texttt{outputs/s1\_eom\_trend\_2015\_2024/}:
\begin{itemize}
    \item \texttt{intra\_month\_slope\_summary.csv} -- Mean/median slopes (Table \ref{tab:slope_summary})
    \item \texttt{intra\_month\_regression.csv} -- Time trend coefficients (Table \ref{tab:time_trend})
    \item \texttt{eom\_level\_regressions.csv} -- Offset-specific trends (Table \ref{tab:offset_trends}, Appendix Table B.1)
\end{itemize}

\textbf{Data Sources.} Minute-level OHLCV data for CME HG futures contracts (2015--2024) are proprietary and not publicly redistributable. The analysis uses expiry metadata from CME Group official contract specifications and the CME Globex holiday calendar (\texttt{config/cme\_holidays.csv}).

\end{document}
